\documentclass{bilidoc}

\title{\textbf{\href{http://dx.doi.org/10.1139/cgj-2016-0037}{Correlations for undrained shear strength of Finnish soft clays.}\\芬兰软黏土不排水剪切强度的相关性}}
\date{\today}
\author{Marco D’Ignazio\thanks{
    \textbf{M. D’Ignazio* and T.T. Länsivaara.}Department of Civil Engineering, Tampere University of Technology, Korkeakoulunkatu 5, 33720, Tampere, Finland.
} \and Kok-Kwang Phoon\thanks{
    \textbf{K.-K. Phoon and S.A. Tan.}Department of Civil and Environmental Engineering, National University of Singapore, No. 1 Engineering Drive 2, 117576, Singapore.
} \and Siew Ann Tan \and Tim Tapani Länsivaara\thanks{
    \textbf{Corresponding author:} Marco D’Ignazio (emails: \url{marco.dignazio@tut.fi}; \url{marco.dignazio@ngi.no}).\newline{}
    \indent{}~*Present address: Norwegian Geotechnical Institute (NGI), Sognsveien 72, N-0855 Oslo, Norway.\newline{}
    \indent{}~Copyright remains with the author(s) or their institution(s). Permission for reuse (free in most cases) can be obtained from \href{https://www.nrcresearchpress.com/page/authors/services/reprints}{RightsLink}.\newline{}\newline{}
    \indent{}~Can. Geotech. J. 53: 1628–1645 (2016) \href{http://dx.doi.org/10.1139/cgj-2016-0037}{dx.doi.org/10.1139/cgj-2016-0037}
}}

\begin{document}

\maketitle

\section*{Abstract 摘要}

\begin{Parallel}{0.60\textwidth}{}
    \ParallelLText
    {
        The study focuses on the derivation of transformation models for undrained shear strength ($s_u$) of Finnish soft sensitive clays. Specific correlation equations for $s_u$ of Finnish clays are presented in this work for the first time. Field and laboratory measurements from 24 test sites in Finland are exploited for this purpose and a multivariate database is constructed. The multivariate data consist of $s_u$ from the field vane test, preconsolidation stress, vertical effective stress, liquid limit, plastic limit, natural water content, and sensitivity. The main objective is to evaluate the interdependence of $s_u$, consolidation stresses, and index parameters and provide a consistent framework for practical use. The new correlations are established through regression analyses. The constructed framework is further validated by another independent multivariate database of clays from Sweden and Norway as well as by empirical equations for Swedish and Norwegian clays. Existing correlations are evaluated for Finnish and Scandinavian clays. Finally, bias and uncertainties of the new correlations are presented.

        \textbf{Key words: }global transformation models, soft clays, multivariate database, undrained shear strength.
    }
    \ParallelRText
    {
        这项研究着重于芬兰软黏土不排水抗剪强度($s_u$)转换模型的推导。芬兰黏土的特殊的相关方程是第一次在这项工作中提出。为此,利用了芬兰24个测试点的现场和实验室测量值,并建立了一个多元数据库。多元数据由现场叶片试验,预固结应力,垂直有效应力,液体极限,塑性极限,天然水含量和敏感性组成。主要目的是评估$s_u$,固结应力和指标参数的相互依赖性,并为实际使用提供一致的框架。通过回归分析建立新的相关性。另一个来自瑞典和挪威的黏土的独立多元数据库以及瑞典和挪威黏土的经验公式进一步验证了构建的框架。对芬兰和斯堪的纳维亚黏土的现有相关性进行评估。最后,提出了新的相关性的偏差和不确定性。

        \textbf{关键词:}全局转换模型,软黏土,多元数据库,不排水剪切强度。
    }
\end{Parallel}

\begin{ParaColumn}[\section{Introduction 介绍}]

    Soft sensitive clays are widespread in Scandinavia, especially on coastal areas. The high compressibility of these soils, along with their low undrained shear strength ($_su$) (even lower than 10 kPa near the ground surface), makes geotechnical design often rather challenging. Therefore, $s_u$ needs to be carefully evaluated for a reliable assessment of the safety level.

    \switchcolumn

    柔软的黏土在斯堪的纳维亚半岛很普遍,特别是在沿海地区。 这些土壤的高可压缩性以及较低的不排水抗剪强度($s_u$)(甚至在地表附近甚至低于10 kPa),使得岩土工程设计通常颇具挑战性。 因此,需要对$s_u$进行仔细评估,以对安全级别进行可靠评估。

    \switchcolumn*

    Scandinavian soft clays are typically slightly over consolidated. The overconsolidation is normally the result of the aging process (e.g., \citealt{Bjerrum19721}). For quick clays, the remolded undrained shear strength ($s_u^{re}$) can be even less than 0.5 kPa and 50–100 times lower than the initially “intact” $_su$ (e.g., \citealt{Rankka2004,Karlsrud20131273}).

    \switchcolumn

    斯堪的纳维亚软黏土通常略有过度固结。 过度固结通常是老化过程的结果(例如\citealt{Bjerrum19721})。 对于快黏土,重塑后的不排水抗剪强度($s_u^{re}$)甚至可以小于0.5 kPa,比最初的“完整” $s_u$低50-100倍(例如,\citealt{Rankka2004,Karlsrud20131273})。

    \switchcolumn*

    $s_u$ can be evaluated from in situ as well as laboratory tests. In Scandinavia, the field vane (FV) test and piezocone cone penetration (CPTU) test are the most commonly used in situ tests. Laboratory tests include undrained triaxial compression (TXC) and direct simple shear (DSS) tests. For some special cases where $s_u$ anisotropy needs to be assessed, triaxial extension (TXE) tests are also performed.

    \switchcolumn

    可以从原位以及实验室试验中评估$s_u$。 在斯堪的纳维亚半岛,现场十字板剪切试验(FV)和带孔压的静力触探试验(CPTU)是最常用的方法。 实验室试验包括不排水三轴压缩(TXC)和直接单剪试验(DSS)试验。 对于某些需要评估$s_u$各向异性的特殊情况,还执行三轴拉伸(TXE)试验。

    \switchcolumn*

    In situations where $s_u$ is not directly measured or measurements are considered to be unreliable, $s_u$ is commonly evaluated from transformation  models  based  on  clay  properties,  such  as vertical preconsolidation pressure (Cpl ) (e.g., \citealt{Mesri1975409,Jamiolkowski198557}) or plasticity (e.g., \citealt{Hansbo1957,Chandler198813}). Such transformation models are typically empirical or semi-empirical, obtained by data fitting through regression analyses (e.g., \citealt{Kulhawy1990}). However, such models must be carefully applied and their limitations be recognized, as soil properties, soil behavior, and site geology may differ from the data source from where the transformation models are calibrated. As a direct consequence, predictions from these models may result in biases with respect to the actual property ($s_u$) values.

    \switchcolumn

    在无法直接测量$s_u$或认为测量结果不可靠的情况下,通常从基于黏土特性(例如垂直预固结压力(Cpl),例如\citealt{Mesri1975409,Jamiolkowski198557})或可塑性的转化模型中评估$s_u$ (例如,\citealt{Hansbo1957,Chandler198813})。 这种转换模型通常是经验的或半经验的,通过回归分析的数据拟合获得(例如,\citealt{Kulhawy1990})。 但是,必须谨慎应用此类模型并认识到其局限性,因为土壤特性,土壤行为和站点地质可能与校准转换模型的数据源不同。 直接的结果是,来自这些模型的预测可能导致相对于实际属性($s_u$)值的偏差。

    \switchcolumn*

    According to \citet{Phoon1999612}, uncertainty coming from transformation models can be customarily categorized as epistemic, meaning that it can be reduced by collecting a greater number of data or improving the available models. Therefore, “global” models, calibrated from data sets covering several sites and soil types, are preferred to “site-specific” models, which are restricted to a specific soil type or a specific site. \citet{Ching201252, Ching2012522, Ching2014663, Ching2014686} presented global models based on soil data covering a large number of test sites from several countries. \citet{Ching201252} pointed out how site-specific models are more accurate (or less uncertain) than global models, although bias can be significant when applied to another site. Instead, global models are less biased, although less precise (or more uncertain).

    \switchcolumn

    根据\citet{Phoon1999612},来自转换模型的不确定性通常可以归类为与认知相关,这意味着可以通过收集更多数据或改进可用模型来减少不确定性。 因此,从涵盖多个地点和土壤类型的数据集校准的“全局”模型优于“地点特定”模型,后者仅限于特定土壤类型或特定地点。 \citet{Ching201252, Ching2012522, Ching2014663, Ching2014686}提出了基于土壤数据的全球模型,该数据涵盖了来自多个国家的大量试验地点。 \citet{Ching201252}指出,尽管在应用于其他站点时偏差可能会很明显,但特定于站点的模型如何比全局模型更准确(或不确定性更低)。 取而代之的是,尽管精度较低(或更不确定),但全局模型的偏差较小。

    \switchcolumn*

    Global transformation models for su of Swedish and Norwegian clays are available in the literature \citep{Larsson1991,Larsson2007,Karlsrud20131273}. How- ever, a comparable model calibrated using a sufficiently large soil research over the last decades, because of its practical database containing Finnish soft clay data is still missing. Therefore, the main objectives of the present paper are (i) to test existing transformation models for $s_u$ for Finnish soft clays and (ii) to derive, for the first time, transformation models for $s_u$ specific to Finnish soft clays using a large multivariate database consisting of FV data points from Finland. Another independent multivariate database of FV data points from Sweden and Norway is compiled and used for comparison and validation.

    \switchcolumn

    瑞典和挪威黏土的全球转换模型可在文献中找到\citep{Larsson1991,Larsson2007,Karlsrud20131273}。 但是,由于缺少包含芬兰软黏土数据的实用数据库,因此在过去几十年中使用足够大的土壤研究进行了校准的可比较模型仍然缺失。 因此,本文的主要目标是(i)测试现有的芬兰软黏土的$s_u$转换模型,以及(ii)首次使用大型多元数据库导出特定于芬兰软黏土的$s_u$转换模型。 由来自芬兰的FV数据点组成。 来自瑞典和挪威的FV数据点的另一个独立的多元数据库已被编译并用于比较和验证。

    \switchcolumn*

    The value of multivariate soil databases has been demonstrated by \citet{Ching201252,Ching2012522, Ching2013907,Ching2014663,Ching2014686} and \citet{Ching201477}. Müller (2013), \citet{Müller2014231,Müller2016603} and \citet{Prästings20161} have demonstrated how uncertainties related to su can be reduced when multivariate soil data are available, showing the benefits of using multivariate analyses (e.g., \citealt{Ching201016})in geotechnical engineering applications. Multivariate soil data-bases are, however, limited in the literature. A summary is given in \enautoref{table:1}. \citet{Ching2014663} proposed labeling a multivar-iate database as “soil type”/“number of parameters of interest”/ “number of data points”. Based on this nomenclature, the two databases presented in this paper are (i) F-CLAY/7/216 for Finnish clays (where “F” stands for Finland) and (ii) S-CLAY/7/168 for Scan-dinavian clays (where “S” stands for Scandinavia). The seven parameters in these databases consisted of $s_u$ from the FV test ($s_U^{\rm{FV}}$), effective vertical stress ($\sigma_v'$), vertical preconsolidation pressure ($\sigma_p'$), natural water content ($w$), liquid limit (LL), plastic limit (PL),and sensitivity ($S_t=s_u/s_u^{re}$).

    \switchcolumn

    \citet{Ching201252,Ching2012522, Ching2013907,Ching2014663,Ching2014686}和\citet{Ching201477}证明了多元土壤数据库的价值。\citet{Müller2014231,Müller2016603}和\citet{Prästings20161}证明了当获得多变量土壤数据时如何减少与$s_u$有关的不确定性,显示了在土力工程应用中使用多变量分析的好处(例如\citealt{Ching201016})。然而,多元土壤数据库在文献中受到限制。\cnautoref{table:1}给出了摘要。\citet{Ching2014663}建议将一个多元数据库标记为“土壤类型” /“感兴趣参数的数量” /“数据点的数量”。基于此术语,本文介绍的两个数据库是(i)芬兰黏土的F-CLAY/7/216(其中“F”代表芬兰)和(ii)斯堪的纳维亚语的S-CLAY/7/168黏土(其中“S”代表斯堪的纳维亚半岛)。这些数据库中的七个参数包括FV试验中的$s_u$($s_U^{\rm{FV}}$),有效垂直应力($\sigma_v'$),垂直预固结压力($\sigma_p'$),天然水含量($w$),液体极限(LL),塑性极限(PL),和灵敏度($S_t=s_u/s_u^{re}$)。

    \begin{table*}[!htb]
    \centering
    \footnotesize
    \caption{Summary of multivariate clay databases.}
    \addtocounter{table}{-1}
    \vspace{-8pt}
    \renewcommand{\tablename}{表}
    \caption{多元黏土数据库概况。}
    \vspace{4pt}
    \renewcommand{\tablename}{Table}
    \setlength{\tabcolsep}{1mm}{
    \begin{tabularx}{\textwidth}{XlllllXX}
        \toprule
        \multirow{2}{*}{Database}  & \multirow{2}{*}{Reference}  &  \multirow{2}{*}{Parameters of interest}  &  \multirow{2}{*}{\shortstack{No. of \\data \\points}}    &  \multirow{2}{*}{\shortstack{No. of \\sites or \\studies}}    & \multicolumn{3}{l}{Range of Properties} \\
             &         &     	 &     &     & OCR   & PI      & $S_t$ \\
        \midrule
        CLAY/5/345      & \citet{Ching2012522}  & $\rm{LI},s_u,s_u^{re},\sigma_p',\sigma_v'$        									    & 345   		    &   37 sites    		 & 1$\sim$4   & -     			& \tabincell{l}{Sensitive to \\quick clays} \\
        \specialrule{0em}{2pt}{2pt}
        CLAY/7/6310     & \citet{Ching2013907}  & \tabincell{l}{$s_u~\rm{under~seven~different~}s_u~$\\$\rm{test~types}$}      								    & 6310  		    &   164 studies   		 & 1$\sim$10  & \tabincell{l}{Low to very \\high plasticity}  & \tabincell{l}{Insensitive to \\quick clays} \\
        \specialrule{0em}{2pt}{2pt}
        CLAY/6/535      & \citet{Ching201477}   & \tabincell{l}{$s_u/\sigma_v',\rm{OCR},(q_t-s_v)/\sigma_v',$\\$(q_t-u_2)/\sigma_v',(u_2-u_0)/\sigma_v',B_q$}      				    & 535   		    &   40 sites    		 & 1$\sim$6   & \tabincell{l}{Low to very \\high plasticity}  & \tabincell{l}{Insensitive to \\quick clays} \\
        \specialrule{0em}{2pt}{2pt}
        CLAY/10/7490    & \citet{Ching2014663}        & \tabincell{l}{$\rm{LL},\rm{PI},\rm{LI},\sigma_v'/P_a,\sigma_p'/P_a,$\\$s_u/\sigma_v',S_t,(q_t-\sigma_v)/\sigma_v',$\\$(q_t-u_2)/\sigma_v',B_q$}      & 7490  		    &   251 studies  		 & 1$\sim$10  & \tabincell{l}{Low to very \\high plasticity}  & \tabincell{l}{Insensitive to \\quick clays} \\
        \bottomrule
    \end{tabularx}}%
    \label{table:1}%
\end{table*}

    \switchcolumn*

    The paper is organized as follows. Firstly, after a brief overview on existing transformation models for $s_u$, the compilation of F-CLAY/7/216 and S-CLAY/7/168 databases is presented. Secondly, 10 dimensionless parameters are derived from the seven basic parameters, resulting in two dimensionless databases. These dimensionless databases (labelled as F-CLAY/10/216 and S-CLAY/10/168) are compared to existing correlations in the literature. To develop more refined correlations for Finnish clays, outliers are removed from F-CLAY/10/216 according to systematic criteria based on soil nature, mechanical characteristics, and statistical considerations. New transformation models for su specific to Finnish clays are derived through regression analyses from the resulting F-CLAY/10/173 database. These new transformation models are compared with existing correlations for Scandinavian clays from the literature. Finally, the performance of the new models derived from F-CLAY/10/173 is evaluated by calculating the biases and uncertainties associated with S-CLAY/10/168.

    \switchcolumn

    本文的结构如下。首先,在对$s_u$的现有转换模型进行简要概述之后,介绍了F-CLAY/7/216和S-CLAY/7/168数据库的编译。其次,从七个基本参数中导出10个无量纲参数,从而形成两个无量纲数据库。将这些无量纲数据库(标记为F-CLAY/10/216和S-CLAY/10/168)与文献中现有的相关性进行了比较。为了建立更精确的芬兰黏土相关性,根据土壤性质,机械特性和统计考虑因素,根据系统标准从F-CLAY/10/216中删除异常值。通过对所得F-CLAY/10/173数据库进行回归分析得出了特定于芬兰黏土的新转换模型。从文献中将这些新的转换模型与斯堪的纳维亚黏土的现有相关性进行了比较。最后,通过计算与S-CLAY/10/168相关的偏差和不确定性来评估源自F-CLAY/10/173的新模型的性能。

\end{ParaColumn}
\begin{ParaColumn}[\section{Overview on existing transformation models for undrained shear strength 不排水抗剪强度的现有转换模型概述}]

    The dependency of $s_u$ on $\sigma_p'$ and plasticity has been the object of research over the last decades, because of its practical usefulness. \citet{Skempton195419} suggested a linear correlation between the normalized $s_u$ determined from FV test ($s_u^{\rm{FV}}$) and plasticity index (PI) for normally consolidated clays. Subsequently, \citet{Chandler198813} indicated that the same correlation could be valid also for overconsolidated clays as shown in \enautoref{equation:1}, although attention must be paid when dealing with fissured, organic, sensitive or other special clays.

    \switchcolumn

    $s_u$对$\sigma_p'$和可塑性的依赖性在过去几十年中一直是研究的对象,因为它具有实用性。 \citet{Skempton195419}提出,通过FV试验确定的归一化$s_u$($s_u^{\rm{FV}}$)与正常固结黏土的可塑性指数(PI)之间存在线性关系。 随后,\citet{Chandler198813}指出,对于\cnautoref{equation:1}中所示的超固结黏土,同样的相关性也可能有效,尽管在处理裂隙,有机,敏感或其他特殊黏土时必须注意。

    \Equation{
        \begin{align}
            \dfrac{s_u^{\rm{FV}}}{\sigma_p'}\approx{}0.11+0.0037\rm{PI}
            \label{equation:1}
        \end{align}
    }
    \switchcolumn*
    
    The dependency of $s_u$ on $\sigma_p'$ and plasticity has been the object of research over the last decades, because of its practical usefulness. \citet{Skempton195419} suggested a linear correlation between the normalized $s_u$ determined from FV test ($s_u^{\rm{FV}}$) and plasticity index (PI) for normally consolidated clays. Subsequently, \citet{Chandler198813} indicated that the same correlation could be valid also for overconsolidated clays as shown in \enautoref{equation:1}, although attention must be paid when dealing with fissured, organic, sensitive or other special clays.

    \switchcolumn

    $s_u$对$\sigma_p'$和可塑性的依赖性在过去几十年中一直是研究的对象,因为它具有实用性。 \citet{Skempton195419}提出,通过FV试验确定的归一化$s_u$($s_u^{\rm{FV}}$)与正常固结黏土的可塑性指数(PI)之间存在线性关系。 随后,\citet{Chandler198813}指出,对于\cnautoref{equation:1}中所示的超固结黏土,同样的相关性也可能有效,尽管在处理裂隙,有机,敏感或其他特殊黏土时必须注意。

    \switchcolumn*

    The dependency of $s_u$ on $\sigma_p'$ and plasticity has been the object of research over the last decades, because of its practical usefulness. \citet{Skempton195419} suggested a linear correlation between the normalized $s_u$ determined from FV test ($s_u^{\rm{FV}}$) and plasticity index (PI) for normally consolidated clays. Subsequently, \citet{Chandler198813} indicated that the same correlation could be valid also for overconsolidated clays as shown in \enautoref{equation:1}, although attention must be paid when dealing with fissured, organic, sensitive or other special clays.

    \switchcolumn

    $s_u$对$\sigma_p'$和可塑性的依赖性在过去几十年中一直是研究的对象,因为它具有实用性。 \citet{Skempton195419}提出,通过FV试验确定的归一化$s_u$($s_u^{\rm{FV}}$)与正常固结黏土的可塑性指数(PI)之间存在线性关系。 随后,\citet{Chandler198813}指出,对于\cnautoref{equation:1}中所示的超固结黏土,同样的相关性也可能有效,尽管在处理裂隙,有机,敏感或其他特殊黏土时必须注意。

    \switchcolumn*

    \citet{Hansbo1957}suggested, for Scandinavian clays, that $s_u^{\rm{FV}}/\sigma_p'$ is directly proportional to LL. \citet{Larsson1980591}, collected strength data points from FV test in Scandinavian clays and proposed a transformation model similar to \enautoref{equation:1}, as described by \enautoref{equation:2}

    \switchcolumn

    \citet{Hansbo1957}提出,对于斯堪的纳维亚黏土,$s_u^{\rm{FV}}/\sigma_p'$与LL成正比。\citet{Larsson1980591}从斯堪的纳维亚黏土的FV试验中收集了强度数据点,并提出了一个与\cnautoref{equation:1}类似的转换模型,如\cnautoref{equation:2}所述。

    \Equation{
        \begin{align}
            \dfrac{s_u^{\rm{FV}}}{\sigma_p'}\approx{}0.08+0.0055\rm{PI}
            \label{equation:2}
        \end{align}
    }
    \switchcolumn*

    According to \citet{Bjerrum19721}, $s_u^{\rm{FV}}$ needs to be converted into mobilized $s_u$($s_u(\rm{mob})\approx{}s_u^{\rm{FV}}\lambda$). The parameter $\lambda$ is a correction multiplier that accounts for rate effects as well as anisotropy, and it is thought to be dependent on the plasticity of the clay.
    
    \citet{Mesri1975409,Mesri1989162} suggested a unique relationship for $s_u(\rm{mob})$ of clays and silts, corresponding approximately to DSS condition (\enautoref{equation:3}), regardless of the plasticity of the clay.

    \switchcolumn

    根据\citet{Bjerrum19721}的说法,$s_u^{\rm{FV}}$需要转换扰动的$s_u$($s_u(\rm{mob})\approx{}s_u^{\rm{FV}}\lambda$)。参数$\lambda$是一个校正倍数,它考虑了速率效应以及各向异性,并且被认为取决于黏土的可塑性。
    
    \citet{Mesri1975409,Mesri1989162}提出黏土和粉砂的$s_u(\rm{mob})$有独特的关系,与黏土的可塑性无关,大约相当于DSS条件(\cnautoref{equation:3})。

    \Equation{
        \begin{align}
            \dfrac{s_u(\rm{mob})}{\sigma_p'}\approx{}0.22
            \label{equation:3}
        \end{align}
    }
    \switchcolumn*
    
    However, according to \citet{Larsson1980591}, \enautoref{equation:3} tends to overestimate $s_u$ in very low–plastic clays, while it underestimates $s_u$ in high-plastic clays. For low overconsolidated clays with low to moderate PI, \citet{Jamiolkowski198557}recommended (\enautoref{equation:4})

    \switchcolumn

    但是,根据\citet{Larsson1980591},\cnautoref{equation:3}倾向于高估低塑性黏土中的$s_u$,而低估了高塑性黏土中的$s_u$。 对于具有低至中等PI的低超固结黏土,\citet{Jamiolkowski198557}推荐(\cnautoref{equation:4})

    \Equation{
        \begin{align}
            \dfrac{s_u(\rm{mob})}{\sigma_v'}\approx{}(0.23\pm{}0.04)\rm{OCR}^{0.8}
            \label{equation:4}
        \end{align}
    }
    \switchcolumn*
    
    The transformation model suggested by \citet{Jamiolkowski198557} is based on the stress history and normalized soil engineering properties (SHANSEP) framework (\enautoref{equation:5}) proposed by \citet{Ladd1974763}. The SHANSEP framework is normally adopted to describe the variation of $s_u$ with the overconsolidation ratio, OCR($=\sigma_p'/\sigma_v'$).

    \switchcolumn

    \citet{Jamiolkowski198557}提出的转换模型基于\citet{Ladd1974763}提出的应力历史和规范化土壤工程特性(SHANSEP)框架(\cnautoref{equation:5})。通常采用SHANSEP框架描述$s_u$随超固结比OCR($=\sigma_p'/\sigma_v'$)的变化。

    \Equation{
        \begin{align}
            \dfrac{s_u}{\sigma_v'}=S(\rm{OCR}^m)
            \label{equation:5}
        \end{align}
    }
    \switchcolumn*

    \noindent{}where $S$ and $m$ are constants dependent on material and test type. $S$ represents the normalized $s_u$ for normally consolidated state. The exponent m varies between 0.75 and 0.95 \citealp{Jamiolkowski198557}. A value of m equal to 0.8 is often assumed in practice. notethat $m = 1$ would reduce \enautoref{equation:5} to eq (3) with $S = 0.22$.

    \switchcolumn

    \noindent{}其中$S$和$m$是取决于材料和试验类型的常数。 $S$表示正常合并状态的归一化$s_u$。 指数$m$在0.75至0.95之间变化\citealp{Jamiolkowski198557}。 在实践中通常假定$m$等于0.8。 注意,当$S=0.22$时,$m = 1$会将\cnautoref{equation:5}减小为\cnautoref{equation:3}。

    \switchcolumn*

    studied the SHANSEP relation between su/v′and OCR for inorganic Scandinavian clays. Data from undrainedTXC, DSS, and TXE tests were collected to assesssuanisotropy. Byassuming an averagemvalue equal to 0.8, it was shown how thenormally consolidated undrained shear strength ratio (S) is material dependent for DSS (eq (6)) and TXE, as it increases with in-creasing liquid limit; while it seems fairly constant for TXC.
    
    Karlsrud and Hernandez-Martinez (2013)studied the (su/v′)–OCR relation for Norwegian soft clays from laboratory tests on high-quality block samples. Outcomes from this study indicate that $s_u$ strongly correlates with natural water content ($w$) combined with OCR (\enautoref{equation:7} for DSS strength). More specifically, peak strengths from TXC, DSS, and TXE test were observed to increase with increasingw. Possible reasons to explain this might be e.g., (i) the open structure typical of Norwegian clays \citealp{Rosenqvist1953195,Rosenqvist1966445}, which allows the soil to retain a quantity of pore water, typically above the liquid limit of the soil or (ii) the increasing rate effects with plasticity.

    \switchcolumn

    研究了无机斯堪的纳维亚黏土中$s_u/\sigma_v'$与OCR之间的SHANSEP关系。从不排水的TXC,DSS和TXE试验中收集数据,以评估各向异性。假设平均值为0.8,表明随着固液极限的增加,正常固结不排水抗剪强度比(S)对于DSS(\cnautoref{equation:6})和TXE的影响与材料有关; \citet{Karlsrud20131273}通过高质量块状样品的实验室试验研究了挪威软黏土的$(s_u/\sigma_v')$-OCR关系。这项研究的结果表明,它与天然水含量($w$)和OCR(DSS强度的\cnautoref{equation:7})密切相关。更具体地说,观察到来自TXC,DSS和TXE试验的峰值强度随$w$的增加而增加。解释这种情况的可能原因可能是,例如(i)挪威黏土\citealp{Rosenqvist1953195,Rosenqvist1966445}的典型的笔形结构,它可以使土壤保留一定数量的孔隙水,通常高于土壤的液位极限;或(ii)增速与可塑性的影响。

    \Equation{
        \begin{align}
            \dfrac{s_u^{\rm{DSS}}}{\sigma_v'}&\approx{}(0.125+0.205\rm{LL}/1.17)\rm{OCR}^{0.8}
            \label{equation:6}\\
            \dfrac{s_u^{\rm{DSS}}}{\sigma_v'}&\approx{}(0.14+0.18w)\rm{OCR}^{0.35+0.77w}
            \label{equation:7}
        \end{align}
    }
    \switchcolumn*

    \citet{Ching2012522} proposed a global transformation model for $s_u(\rm{mob})$ from FV and unconfined compression (UC) tests as a function of OCR and $S_t$. The model was built based on a large database of structured clays (CLAY/5/345) consisting of 345 clay data points from several locations all over the world (\enautoref{equation:8}).

    \switchcolumn

    \citet{Ching2012522}提出了FV和无侧限压缩(UC)试验的$s_u(\rm{mob})$全局转换模型,该模型是OCR和$S_t$的函数。该模型是基于大型结构性黏土数据库(CLAY/5/345)由来自世界各地的345个黏土数据点组成(\cnautoref{equation:8})。

    \Equation{
        \begin{align}
            \dfrac{s_u(\rm{mob})}{\sigma_v'}\approx{}0.229\rm{OCR}^{0.823}S_t^{0.121}
            \label{equation:8}
        \end{align}
    }

\end{ParaColumn}

\begin{appendix}
\end{appendix}

\bibliographystyle{plainnat} % gbt7714-author-year gbt7714-numerical
\bibliography{DIgnazio2016.bib}

\end{document}