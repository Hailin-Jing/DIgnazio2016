\begin{ParaColumn}[\section{$s_u/\sigma_v^\prime$ transformation models for F-CLAY/10/173 \\F-CLAY/10/173数据库的$s_u/\sigma_v^\prime$转换模型}]
    \switchcolumn[0]*[\Paragraph{Removal of outliers in F-CLAY/10/216 移除F-CLAY/10/216数据库中的异常值}]

    As the scope of this study is to derive transformation models for $s_u$ of Finnish soft clays that are more refined than the existing models in the literature, the data points collected in F-CLAY/10/216 are analyzed with the purpose of improving the quality of the database by removing outliers. The quality of data points is assessed through criteria based on the physical nature of the soil, mechanical characteristics, and statistical considerations. The adopted criteria are listed below:

    \switchcolumn

    由于本研究的范围是获得比文献中现有模型更精细的芬兰软黏土$s_u$的转化模型,因此对F-CLAY/10/216中收集的数据点进行了分析,目的在于通过消除异常值来提高数据库的质量。 通过基于土壤物理性质,机械特性和统计考虑因素的标准评估数据点的质量。 通过的数据标准如下:

    \switchcolumn*

    \begin{enumerate}
        \item Points located near the ground surface that may belong to fissured upper layers (dry crust), as the study focuses on intact and saturated clays. Dry crust layers are generally unsaturated and contain cracks and fissures. $s_u$ of such soils may be highly overestimated when measured with the FV test \citep{LaRochelle1974142,Lefebvre198723,D'Ignazio20153639}. Dry crust layers in Finland are normally 1–2 m thick. Therefore, points near the ground surface, at depths lower than 1.50 m, are left out.\\
        
        \item Points with $s_{\mathrm{u}}(\mathrm{mob})/\sigma_{\mathrm{p}}^{\prime}$ lower than an initial shear stress mobilization ($\tau_{0}/\sigma_{\mathrm{p}}^{\prime}$ where $\tau_{0}$ is the initially mobilized shear stress) in the soil $\tau_{\mathrm{o}} / \sigma_{\mathrm{p}}^{\prime}=0.5\left(1-K_{0}\right)$ equal to 0.15 for normally consolidated state. $K_{0}$ is the earth pressure coefficient at rest calculated from \citet{Jaky1944355} formula ($K_{0}=1-\sin \phi^{\prime}$, where $\phi^{\prime}$ is the effective friction angle of the soil). $\tau_{0}=0.15$ implies $\phi^{\prime}=18^{\circ}$ which could represent, according to the authors' experience, the lowest boundary value for friction angle of Scandinavian clays.
        
        \item Outliers identified through the "$2\sigma$" ($95\%$ confidence interval of $s_{\mathrm{u}}(\mathrm{mob}) / \sigma_{\mathrm{v}}^{\prime}$) statistical criteria. "$\sigma$" is the standard deviation of the variable $s_{\mathrm{u}}(\mathrm{mob})/\sigma_{\mathrm{v}}^{\prime}$. Data points where, for a given "$i$" value $\left|\left[s_{\mathrm{u}}(\mathrm{mob}) / \sigma_{\mathrm{v}}^{\prime}\right]_{i}-\operatorname{mean}\left[s_{\mathrm{u}}(\mathrm{mob}) / \sigma_{\mathrm{v}}^{\prime}\right]\right|>2\sigma$, are removed. Normally, outliers for a given data set are identified using the $3 \sigma$ (three sigma) rule, representing the $99\%$ confidence interval of the data. The reason why in this study the $95 \%$ confidence interval criteria is used, has to do with the inherent soil variability. $s_{\mathrm{u}}$ profiles obtained from the FV test are likely to show clear fluctuations against a mean trend. Strength variability with depth may depend not only on the consolidation stresses (initial or mechanically induced), but also on the inherent variability of the soil layers (variation of grain size, index properties). Furthermore, sample disturbance can affect the preconsolidation pressure $\left(\sigma_{\mathrm{p}}^{\prime}\right)$ trend with depth and consequently the ratio $s_{\mathrm{u}}(\mathrm{mob})/\sigma_{\mathrm{p}}^{\prime} .$ To remove these points, a statistical criterion stronger than the " $3 \sigma$ " was preferred to a "visual" one.
    \end{enumerate}

    \switchcolumn

    \begin{enumerate}
        \item 由于研究重点是完整的和饱和的黏土,因此位于地面附近的点可能属于裂隙的上层(干硬皮)。 干燥的地壳层通常是不饱和的,并且包含裂纹和裂缝。 用FV试验测量时,此类土壤中的$s_u$可能会被高估\citep{LaRochelle1974142,Lefebvre198723,D'Ignazio20153639}。 芬兰的干硬皮层通常为1-2m厚。 因此,在地面附近的深度小于1.50m的点被忽略了。
        
        \item 在正常固结状态下,$s_{\mathrm{u}}(\mathrm{mob})/\sigma_{\mathrm{p}}^{\prime}$低于初始剪应力激发点($\tau_{0}/\sigma_{\mathrm{p}}^{\prime}$,其中$\tau_{0}$是初始动剪应力),$\tau_{\mathrm{o}} / \sigma_{\mathrm{p}}^{\prime}=0.5\left(1-K_{0}\right)$等于0.15。 $K_0$是根据\citet{Jaky1944355}公式计算得出的静止土压力系数($K_{0}=1-\sin \phi^{\prime}$,其中$\phi^{\prime}$是土壤的有效摩擦角)。$\tau_{0}=0.15$表示$\phi^{\prime}=18^{\circ}$,根据作者的经验,这可能代表斯堪的纳维亚黏土的摩擦角最低边界值。
        
        \item 通过“$2\sigma$”($s_{\mathrm{u}}(\mathrm{mob}) / \sigma_{\mathrm{v}}^{\prime}$的95$\%$置信区间)统计标准识别异常值。  “$\sigma$”是变量$s_{\mathrm{u}}(\mathrm{mob})/\sigma_{\mathrm{v}}^{\prime}$的标准偏差。 数据指向给定的“$i$”值,$\left|\left[s_{\mathrm{u}}(\mathrm{mob}) / \sigma_{\mathrm{v}}^{\prime}\right]_{i}-\operatorname{mean}\left[s_{\mathrm{u}}(\mathrm{mob}) / \sigma_{\mathrm{v}}^{\prime}\right]\right|>2\sigma$的点将被移除。 通常,给定数据集的离群值使用$3\sigma$规则来标识,代表数据的$99\%$置信区间。 在本研究中使用$95\%$置信区间标准的原因与土壤固有的变异性有关。 从FV试验获得的su轮廓可能显示出相对于平均趋势的明显波动。 强度随深度的变化可能不仅取决于固结应力(初始应力或机械应力),还取决于土层的固有变化性(晶粒尺寸,指数特性的变化)。 此外,样品扰动会影响预固结压力(�p')随深度变化的趋势,因此会影响su(mob)/�p'之比。 为了消除这些问题,统计标准比“ 3.”强于“视觉”。
    \end{enumerate}
    
\end{ParaColumn}