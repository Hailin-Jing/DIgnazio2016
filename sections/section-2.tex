\begin{ParaColumn}[\section{Overview on existing transformation models for undrained shear strength 不排水抗剪强度的现有转换模型概述}]

    The dependency of $s_u$ on $\sigma_p'$ and plasticity has been the object of research over the last decades, because of its practical usefulness. \citet{Skempton195419} suggested a linear correlation between the normalized $s_u$ determined from FV test ($s_u^{\rm{FV}}$) and plasticity index (PI) for normally consolidated clays. Subsequently, \citet{Chandler198813} indicated that the same correlation could be valid also for overconsolidated clays as shown in \enautoref{equation:1}, although attention must be paid when dealing with fissured, organic, sensitive or other special clays.

    \switchcolumn

    $s_u$对$\sigma_p'$和可塑性的依赖性在过去几十年中一直是研究的对象,因为它具有实用性。 \citet{Skempton195419}提出,通过FV试验确定的归一化$s_u$($s_u^{\rm{FV}}$)与正常固结黏土的可塑性指数(PI)之间存在线性关系。 随后,\citet{Chandler198813}指出,对于\cnautoref{equation:1}中所示的超固结黏土,同样的相关性也可能有效,尽管在处理裂隙,有机,敏感或其他特殊黏土时必须注意。

    \Equation{
        \begin{align}
            \dfrac{s_u^{\rm{FV}}}{\sigma_p'}\approx{}0.11+0.0037\rm{PI}
            \label{equation:1}
        \end{align}
    }
    \switchcolumn*
    
    The dependency of $s_u$ on $\sigma_p'$ and plasticity has been the object of research over the last decades, because of its practical usefulness. \citet{Skempton195419} suggested a linear correlation between the normalized $s_u$ determined from FV test ($s_u^{\rm{FV}}$) and plasticity index (PI) for normally consolidated clays. Subsequently, \citet{Chandler198813} indicated that the same correlation could be valid also for overconsolidated clays as shown in \enautoref{equation:1}, although attention must be paid when dealing with fissured, organic, sensitive or other special clays.

    \switchcolumn

    $s_u$对$\sigma_p'$和可塑性的依赖性在过去几十年中一直是研究的对象,因为它具有实用性。 \citet{Skempton195419}提出,通过FV试验确定的归一化$s_u$($s_u^{\rm{FV}}$)与正常固结黏土的可塑性指数(PI)之间存在线性关系。 随后,\citet{Chandler198813}指出,对于\cnautoref{equation:1}中所示的超固结黏土,同样的相关性也可能有效,尽管在处理裂隙,有机,敏感或其他特殊黏土时必须注意。

    \switchcolumn*

    The dependency of $s_u$ on $\sigma_p'$ and plasticity has been the object of research over the last decades, because of its practical usefulness. \citet{Skempton195419} suggested a linear correlation between the normalized $s_u$ determined from FV test ($s_u^{\rm{FV}}$) and plasticity index (PI) for normally consolidated clays. Subsequently, \citet{Chandler198813} indicated that the same correlation could be valid also for overconsolidated clays as shown in \enautoref{equation:1}, although attention must be paid when dealing with fissured, organic, sensitive or other special clays.

    \switchcolumn

    $s_u$对$\sigma_p'$和可塑性的依赖性在过去几十年中一直是研究的对象,因为它具有实用性。 \citet{Skempton195419}提出,通过FV试验确定的归一化$s_u$($s_u^{\rm{FV}}$)与正常固结黏土的可塑性指数(PI)之间存在线性关系。 随后,\citet{Chandler198813}指出,对于\cnautoref{equation:1}中所示的超固结黏土,同样的相关性也可能有效,尽管在处理裂隙,有机,敏感或其他特殊黏土时必须注意。

    \switchcolumn*

    \citet{Hansbo1957}suggested, for Scandinavian clays, that $s_u^{\rm{FV}}/\sigma_p'$ is directly proportional to LL. \citet{Larsson1980591}, collected strength data points from FV test in Scandinavian clays and proposed a transformation model similar to \enautoref{equation:1}, as described by \enautoref{equation:2}

    \switchcolumn

    \citet{Hansbo1957}提出,对于斯堪的纳维亚黏土,$s_u^{\rm{FV}}/\sigma_p'$与LL成正比。\citet{Larsson1980591}从斯堪的纳维亚黏土的FV试验中收集了强度数据点,并提出了一个与\cnautoref{equation:1}类似的转换模型,如\cnautoref{equation:2}所述。

    \Equation{
        \begin{align}
            \dfrac{s_u^{\rm{FV}}}{\sigma_p'}\approx{}0.08+0.0055\rm{PI}
            \label{equation:2}
        \end{align}
    }
    \switchcolumn*

    According to \citet{Bjerrum19721}, $s_u^{\rm{FV}}$ needs to be converted into mobilized $s_u$($s_u(\rm{mob})\approx{}s_u^{\rm{FV}}\lambda$). The parameter $\lambda$ is a correction multiplier that accounts for rate effects as well as anisotropy, and it is thought to be dependent on the plasticity of the clay.
    
    \citet{Mesri1975409,Mesri1989162} suggested a unique relationship for $s_u(\rm{mob})$ of clays and silts, corresponding approximately to DSS condition (\enautoref{equation:3}), regardless of the plasticity of the clay.

    \switchcolumn

    根据\citet{Bjerrum19721}的说法,$s_u^{\rm{FV}}$需要转换扰动的$s_u$($s_u(\rm{mob})\approx{}s_u^{\rm{FV}}\lambda$)。参数$\lambda$是一个校正倍数,它考虑了速率效应以及各向异性,并且被认为取决于黏土的可塑性。
    
    \citet{Mesri1975409,Mesri1989162}提出黏土和粉砂的$s_u(\rm{mob})$有独特的关系,与黏土的可塑性无关,大约相当于DSS条件(\cnautoref{equation:3})。

    \Equation{
        \begin{align}
            \dfrac{s_u(\rm{mob})}{\sigma_p'}\approx{}0.22
            \label{equation:3}
        \end{align}
    }
    \switchcolumn*
    
    However, according to \citet{Larsson1980591}, \enautoref{equation:3} tends to overestimate $s_u$ in very low–plastic clays, while it underestimates $s_u$ in high-plastic clays. For low overconsolidated clays with low to moderate PI, \citet{Jamiolkowski198557}recommended (\enautoref{equation:4})

    \switchcolumn

    但是,根据\citet{Larsson1980591},\cnautoref{equation:3}倾向于高估低塑性黏土中的$s_u$,而低估了高塑性黏土中的$s_u$。 对于具有低至中等PI的低超固结黏土,\citet{Jamiolkowski198557}推荐(\cnautoref{equation:4})

    \Equation{
        \begin{align}
            \dfrac{s_u(\rm{mob})}{\sigma_v'}\approx{}(0.23\pm{}0.04)\rm{OCR}^{0.8}
            \label{equation:4}
        \end{align}
    }
    \switchcolumn*
    
    The transformation model suggested by \citet{Jamiolkowski198557} is based on the stress history and normalized soil engineering properties (SHANSEP) framework (\enautoref{equation:5}) proposed by \citet{Ladd1974763}. The SHANSEP framework is normally adopted to describe the variation of $s_u$ with the overconsolidation ratio, OCR($=\sigma_p'/\sigma_v'$).

    \switchcolumn

    \citet{Jamiolkowski198557}提出的转换模型基于\citet{Ladd1974763}提出的应力历史和规范化土壤工程特性(SHANSEP)框架(\cnautoref{equation:5})。通常采用SHANSEP框架描述$s_u$随超固结比OCR($=\sigma_p'/\sigma_v'$)的变化。

    \Equation{
        \begin{align}
            \dfrac{s_u}{\sigma_v'}=S(\rm{OCR}^m)
            \label{equation:5}
        \end{align}
    }
    \switchcolumn*

    \noindent{}where $S$ and $m$ are constants dependent on material and test type. $S$ represents the normalized $s_u$ for normally consolidated state. The exponent m varies between 0.75 and 0.95 \citealp{Jamiolkowski198557}. A value of m equal to 0.8 is often assumed in practice. notethat $m = 1$ would reduce \enautoref{equation:5} to eq (3) with $S = 0.22$.

    \switchcolumn

    \noindent{}其中$S$和$m$是取决于材料和试验类型的常数。 $S$表示正常合并状态的归一化$s_u$。 指数$m$在0.75至0.95之间变化\citealp{Jamiolkowski198557}。 在实践中通常假定$m$等于0.8。 注意,当$S=0.22$时,$m = 1$会将\cnautoref{equation:5}减小为\cnautoref{equation:3}。

    \switchcolumn*

    studied the SHANSEP relation between su/v′and OCR for inorganic Scandinavian clays. Data from undrainedTXC, DSS, and TXE tests were collected to assesssuanisotropy. Byassuming an averagemvalue equal to 0.8, it was shown how thenormally consolidated undrained shear strength ratio (S) is material dependent for DSS (eq (6)) and TXE, as it increases with in-creasing liquid limit; while it seems fairly constant for TXC.
    
    Karlsrud and Hernandez-Martinez (2013)studied the (su/v′)–OCR relation for Norwegian soft clays from laboratory tests on high-quality block samples. Outcomes from this study indicate that $s_u$ strongly correlates with natural water content ($w$) combined with OCR (\enautoref{equation:7} for DSS strength). More specifically, peak strengths from TXC, DSS, and TXE test were observed to increase with increasingw. Possible reasons to explain this might be e.g., (i) the open structure typical of Norwegian clays \citealp{Rosenqvist1953195,Rosenqvist1966445}, which allows the soil to retain a quantity of pore water, typically above the liquid limit of the soil or (ii) the increasing rate effects with plasticity.

    \switchcolumn

    研究了无机斯堪的纳维亚黏土中$s_u/\sigma_v'$与OCR之间的SHANSEP关系。从不排水的TXC,DSS和TXE试验中收集数据,以评估各向异性。假设平均值为0.8,表明随着固液极限的增加,正常固结不排水抗剪强度比(S)对于DSS(\cnautoref{equation:6})和TXE的影响与材料有关; \citet{Karlsrud20131273}通过高质量块状样品的实验室试验研究了挪威软黏土的$(s_u/\sigma_v')$-OCR关系。这项研究的结果表明,它与天然水含量($w$)和OCR(DSS强度的\cnautoref{equation:7})密切相关。更具体地说,观察到来自TXC,DSS和TXE试验的峰值强度随$w$的增加而增加。解释这种情况的可能原因可能是,例如(i)挪威黏土\citealp{Rosenqvist1953195,Rosenqvist1966445}的典型的笔形结构,它可以使土壤保留一定数量的孔隙水,通常高于土壤的液位极限;或(ii)增速与可塑性的影响。

    \Equation{
        \begin{align}
            \dfrac{s_u^{\rm{DSS}}}{\sigma_v'}&\approx{}(0.125+0.205\rm{LL}/1.17)\rm{OCR}^{0.8}
            \label{equation:6}\\
            \dfrac{s_u^{\rm{DSS}}}{\sigma_v'}&\approx{}(0.14+0.18w)\rm{OCR}^{0.35+0.77w}
            \label{equation:7}
        \end{align}
    }
    \switchcolumn*

    \citet{Ching2012522} proposed a global transformation model for $s_u(\rm{mob})$ from FV and unconfined compression (UC) tests as a function of OCR and $S_t$. The model was built based on a large database of structured clays (CLAY/5/345) consisting of 345 clay data points from several locations all over the world (\enautoref{equation:8}).

    \switchcolumn

    \citet{Ching2012522}提出了FV和无侧限压缩(UC)试验的$s_u(\rm{mob})$全局转换模型,该模型是OCR和$S_t$的函数。该模型是基于大型结构性黏土数据库(CLAY/5/345)由来自世界各地的345个黏土数据点组成(\cnautoref{equation:8})。

    \Equation{
        \begin{align}
            \dfrac{s_u(\rm{mob})}{\sigma_v'}\approx{}0.229\rm{OCR}^{0.823}S_t^{0.121}
            \label{equation:8}
        \end{align}
    }

\end{ParaColumn}